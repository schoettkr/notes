% Created 2018-04-08 Sun 16:27
% Intended LaTeX compiler: pdflatex
\documentclass[11pt]{article}
\usepackage[utf8]{inputenc}
\usepackage[T1]{fontenc}
\usepackage{graphicx}
\usepackage{grffile}
\usepackage{longtable}
\usepackage{wrapfig}
\usepackage{rotating}
\usepackage[normalem]{ulem}
\usepackage{amsmath}
\usepackage{textcomp}
\usepackage{amssymb}
\usepackage{capt-of}
\usepackage{hyperref}
\usepackage{amssymb}
\DeclareMathOperator*{\argmax}{arg\,max}
\usepackage{istgame}
\DeclareMathOperator*{\argmin}{arg\,min}
\date{\today}
\title{}
\hypersetup{
 pdfauthor={},
 pdftitle={},
 pdfkeywords={},
 pdfsubject={},
 pdfcreator={Emacs 26.0.91 (Org mode 9.1.6)}, 
 pdflang={English}}
\begin{document}

\tableofcontents


\section{Entscheidungstheorie}
\label{sec:org2e17168}
\subsection{Naive Entscheidungsregeln / Entscheidung unter Ungewissheit}
\label{sec:org3ac92b1}
Als Entscheidung unter Ungewissheit bezeichnet man Entscheidungssituationen, bei denen zwar die Alternativen(Strategien), die möglichen Umweltzustände und die Ergebnisse bei Wahl einer bestimmten Alternative und Eintritt eines bestimmten Umweltzustandes bekannt sind, die \emph{Eintrittswahrscheinlichkeiten der Umweltzustände} jedoch \emph{ubekannt} sind.\\
Beispiel Aufgabe: \emph{Person besitzt 100\texteuro{} und muss sich zwischen zwei Strategien s \(\epsilon\) S (= s\(_{\text{1}}\) , s\(_{\text{2}}\)) entscheiden. Es gibt zwei Umweltzustände z \(\epsilon\) Z (= z\(_{\text{1}}\), z\(_{\text{2}}\)). Welche Strategie wählt die Person je nach angewendetem Entscheidungskriterium, wenn sie sich für \textbf{eine} Strategie entscheidet?}\\
\newline
1.) Erstelle Auszahlungsmatrix\\
m = Kapital der Person = 100\texteuro{}\\
\(\pi\) = Auszahlungsfunktion die von Strategie s und Umweltzustand z abhängt und eine dementsprechende Auszahlung angibt: Wenn s\(_{\text{1}}\) für Umweltzustand z\(_{\text{1}}\) eine Steigerung des Vermögens um 40\% garantiert und für Umweltzustand s\(_{\text{2}}\) -20\%, dann betragen die jeweiligen Auszahlungen die in die Auszahlungsmatrix gehören:

\begin{equation*}
\begin{aligned}
\pi(s_1,z_1)=m*(1+0.4)=100*1.4=140\\
\pi(s_1,z_2)=m*(1-0.2)=100*0.8=80
\end{aligned}
\end{equation*}
Für s\(_{\text{2}}\) , bei z\(_{\text{1}}\) -10\% und für z\(_{\text{2}}\) 20\%, betragen die Auszahlungen demnach:
\begin{equation*}
\begin{aligned}
\pi(s_2,z_1)=100*0.9=90\\
\pi(s_2,z_2)=100*(1+0.2)=120
\end{aligned}
\end{equation*}
Auszahlungsmatrix:
\begin{center}
\begin{tabular}{c|c|c}
s\textbackslash{z} & z\(_{\text{1}}\) & z\(_{\text{2}}\)\\
\hline
s\(_{\text{1}}\) & 100*1.4=140 & 100*0.8=80\\
s\(_{\text{2}}\) & 100*0.9=90 & 100*1.2=120\\
\end{tabular}
\end{center}
Bei einer möglichen Mischanlage oder Mischwahl wird die Auszahlungsmatrix um eine Zeile(Mischstrategie erweitert):
\begin{center}
\begin{tabular}{c|c|c}
s\textbackslash{z} & z\(_{\text{1}}\) & z\(_{\text{2}}\)\\
\hline
s\(_{\text{1}}\) & 100*1.4=140 & 100*0.8=80\\
MS & 140*\(\alpha\) + (1-\(\alpha\))*90 & 80*\(\alpha\) + (1-\(\alpha\))*120\\
s\(_{\text{2}}\) & 100*0.9=90 & 100*1.2=120\\
\end{tabular}
\end{center}
\(\alpha\) beschreibt den Anteil der in s\(_{\text{1}}\) investiert wird, demnach fließt 1-\(\alpha\) in s\(_{\text{2}}\) fuer jeden Umweltzustand
\newline
2.) Strategie Wahl\\
Entscheidungsregeln: Maximin, Maximax, Hurwics, Regel des minimalen Bedauerns, Laplace\\
\subsubsection{Maximin (Pessimist)}
\label{sec:orgcf14991}
Immer vom schlechtesten Fall ausgehen(risikoavers). Wähle daher die Strategie s \(\epsilon\) S mit dem \textbf{größtem} Zeilen\textbf{minimum}:\\
\(\displaystyle Zeilenminimum = \min_{z \epsilon Z} \pi(s,z)\)
\newline
\(\displaystyle Zeilenminimum_{s_1} = \min_{z \epsilon Z} \pi(s_1,z)\)
\newline
\(\displaystyle Zeilenminimum_{s_2} = \min_{z \epsilon Z} \pi(s_2,z)\)

Betrachte also fuer jede Strategie alle Auszahlungen fuer die jeweiligen Umweltzustände z \(\epsilon\) Z und suche pro Strategie die kleinste Auszahlung:\\
Fuer s\(_{\text{1}}\) beispielsweise: \(\pi\)(s\(_{\text{1}}\), z\(_{\text{1}}\)) = 140 und \(\pi\)(s\(_{\text{1}}\), z\(_{\text{2}}\)) = 80 \(\rightarrow\) 
\(\displaystyle \min_{z \epsilon Z} \pi(s_1,z) = 80\)
\newline
Und fuer die zweite Strategie: \(\displaystyle \min_{z \epsilon Z} \pi(s_2,z) = 90\)
\newline
Finde nun das Maximum der Zeilenminima der Strategien um die optimale Strategie nach der Maximin-Regel herauszufinden:
\(\displaystyle \argmax_{s \epsilon S}\min_{z \epsilon Z} \pi(s,z) = \argmax_{s \epsilon S}(80,90) = \arg(90) = s_2\)
\newline
\(\rightarrow\) Nach der Maximin-Regel wählt die Person s2, da s2 im "schlimmsten" Falls das "beste" Ergebnis liefert.\\

 \textbf{Mischanlage}\\
Setze MS\(_{\text{z}_{\text{1}}}\) und MS\(_{\text{z}_{\text{2}}}\) gleich und löse nach \(\alpha\) auf:\\
\begin{equation*}
\begin{aligned}
&140*\alpha + (1-\alpha)*90 = 80*\alpha + (1-\alpha)*120\\
&140\alpha - 90\alpha + 90 = 80\alpha - 120\alpha +120\\
&50\alpha + 90 = -40 \alpha + 120\\
&90\alpha = 30\\
&\alpha = \frac{3}{9} = \frac{1}{3}
\end{aligned}
\end{equation*}
Bei Maximin wird \(\alpha = \frac{1}{3}\) gewählt, da dieses \(\alpha\) unabhängig vom eintretenden Umweltzustand für beide Strategien die selbe Auszahlung(einsetzen) ergibt.
\newline
\subsubsection{Maximax (Optimist)}
\label{sec:org3d5c3ad}
Immer vom besten Fall ausgehen(risikofreudig). Wähle daher die Strategie s mit dem \textbf{größtem} Zeilen\textbf{maximum}:\\
\(\displaystyle Zeilenmaximum = \max_{z \epsilon Z} \pi(s,z)\)
\newline
Die optimale Strategie s* ergibt sich nach der Maximax-Regel wie folgt:\\
\(\displaystyle s^* = \argmax_{s \epsilon S} \max_{z \epsilon Z} \pi(s,z)\)
\newline
\textbf{Mischanlage}\\
Da risikofreudig einmal \(\alpha=0\) und \(\alpha=1\) in beide Mischstrategien einsetzen und maximale Auszahlung bei \(\alpha\) = 0 und \(\alpha\) = 1 ermitteln und dann wiederum das Maximum der beiden Maxima ermitteln.\\
\(\alpha\) = 0:\\
\begin{equation*}
\begin{aligned}
&MS_{z_1}=140*0 + (1-0)*90 = 90\\
&MS_{z_2}=80*0 + (1-0)*120 = 120
\end{aligned}
\end{equation*}
\newline
\(\alpha\) = 1:\\
\begin{equation*}
\begin{aligned}
&MS_{z_1}=140*1 + (1-1)*90 = 140\\
&MS_{z_2}=80*1 + (1-1)*120 = 80
\end{aligned}
\end{equation*}
\newline
\(\rightarrow\) \(\max(120,140)=140\) \(\rightarrow\) wähle \(\alpha\) = 1, da dort das größte Maximum möglich ist.
\newline\\
\subsubsection{Hurwics}
\label{sec:org3b252de}
Ist eine Mischform aus Optimismus und Pessimismus und wird daher mithilfe des "Optimismuskoeffizienten" \(\gamma\) berechnet:\\
\(\displaystyle s^* = \argmax_{s \epsilon S}(\gamma * \max_{z \epsilon Z} \pi(s,z) + (1 - \gamma) * \min_{z \epsilon Z} \pi(s,z))\)
\newline
Vorgehen: 
\(\displaystyle \gamma * \max_{z \epsilon Z} \pi(s,z) + (1 - \gamma) * \min_{z \epsilon Z} \pi(s,z)\) für jede Strategie ausfüllen (also den Optimismuskoeffizienten \(\gamma\) mal den "besten" Fall plus 1 minus \(\gamma\) mal den schlechten Fall für jede Strategie):\\
\begin{equation*}
\begin{aligned}
s_1: \gamma * 140 + 1 - \gamma * 80\\
s_2: \gamma * 120 + 1 - \gamma * 90
\end{aligned}
\end{equation*}
Danach beide Formeln gleichsetzen und nach \(\gamma\) auflösen, um das \(\gamma\) zu finden bei dem die erwarteten Auszahlungen gleich und die Person indifferent zwischen beiden Strategien ist
\begin{equation*}
\begin{aligned}
&\gamma * 140 + (1 - \gamma) * 80 = \gamma * 120 + (1 - \gamma) * 90\\
&140\gamma - 80\gamma + 80 = 120\gamma - 90\gamma + 90\\
&60\gamma + 80 = 30\gamma + 90\\
&30\gamma = 10\\
&\gamma = \frac{10}{30} = \frac{1}{3}
\end{aligned}
\end{equation*}
Bei \(\gamma=\frac{1}{3}\) ist die Person indifferent zwischen s\(_{\text{1}}\) und s\(_{\text{2}}\), da bei beiden Strategien eine Auszahlung von 100 zu erwarten ist. Für \(\gamma<\frac{1}{3}\) liefert s\(_{\text{2}}\) höhere Auszahlungen (rausfinden durch einsetzen) und fuer \(\gamma>\frac{1}{3}\) ist s\(_{\text{1}}\) die bessere Wahl.\\
\[ s^* =\begin{cases} 
      s_2 & \gamma < \frac{1}{3} \\
      indifferent & \gamma=\frac{1}{3} \\
      s_1 & \gamma > \frac{1}{3}
   \end{cases}
\]
\newline
\textbf{Mischanlage}\\
Mischung aus Pessimist(bei dem \(\alpha = \frac{1}{3}\)) war und Optimist bei dem \(\alpha\) = 1 war. Mischstrategie für beide Umweltzustände mit beiden \(\alpha\)'s einsetzen. Dann für beide Auszahlungen des jeweiligen Alpha, die höhere Auszahlung mal \(\gamma\) plus die niedrigere Auszahlung mal 1 minus \(\gamma\) und so weit wie möglich ausmultiplizieren.\\
\(\alpha = \frac{1}{3}\): \\
\begin{equation*}
\begin{aligned}
&MS_{z_1}=140*\frac{1}{3} + \frac{2}{3}*90=\frac{320}{3}\\
&MS_{z_2}=80*\frac{1}{3} + \frac{2}{3}*120=\frac{320}{3}\\
\rightarrow \frac{320}{3}*\gamma + (1-\gamma)*\frac{320}{3} = \frac{320}{3}
\end{aligned}
\end{equation*}

\(\alpha = 1\): \\
\begin{equation*}
\begin{aligned}
&MS_{z_1}140*1 + (1-1)*90=140\\
&MS_{z_2}80*1 + (1-1)*120=80\\
\rightarrow 140*\gamma + (1-\gamma)*80 = 60\gamma+80
\end{aligned}
\end{equation*}

Die beiden daraus resultierenden Gleichungen nun gleichsetzen und nach Gamma auflösen:\\
\begin{equation*}
\begin{aligned}
60\gamma+80 = \frac{320}{3}\\
60\gamma = \frac{80}{3}\\
\gamma = \frac{4}{9}
\end{aligned}
\end{equation*}

Bei diesem \(\gamma = \frac{4}{9}\) ist die Person indifferent zwischen Pessimismus (\(\alpha = \frac{1}{3}\)) und Optimismus (\(\alpha\) =1)\\
\[ \alpha =\begin{cases} 
      \frac{1}{3} & \gamma < \frac{4}{9} \\
      {\frac{1}{3},1} & \gamma=\frac{4}{9} \\
      1 & \gamma > \frac{4}{9}
   \end{cases}
\]
\newline\\
\subsubsection{Regel des minimalen Bedauerns}
\label{sec:org77d98fc}
Überführe die Auszahlungsmatrix in eine "Bedauernsmatrix": "Wieviel geht mir durch die Lappen wenn der jeweils schlechtere Zustand eintritt?" \(\rightarrow\) Differenz zum Spaltenmaximum bilden und eintragen und für Spaltenmaximum = 0:\\
\begin{center}
\begin{tabular}{c|c|c}
s\textbackslash{z} & z\(_{\text{1}}\) & z\(_{\text{2}}\)\\
\hline
s\(_{\text{1}}\) & 140 & 80\\
s\(_{\text{2}}\) & 90 & 120\\
\end{tabular}
\end{center}
wird zur Bedauernsmatrix:
\begin{center}
\begin{tabular}{c|c|c}
s\textbackslash{z} & z\(_{\text{1}}\) & z\(_{\text{2}}\)\\
\hline
s\(_{\text{1}}\) & 0 & 40\\
s\(_{\text{2}}\) & 50 & 0\\
\end{tabular}
\end{center}
Wähle dann die Strategie mit dem geringsten Bedauern, also die Strategie mit dem kleinsten Zeilenmaximum\\
\(\displaystyle s^* = \argmin_{s \epsilon S}\max_{z \epsilon Z} = \argmin_{s \epsilon S}(40, 50) = \arg(40) = s_1\)
\newline\\
\textbf{Mischanlage}\\
Ziehe die Mischstrategie vom Spaltenmaximum ab (für alle Umweltzustände) um das Bedauern festzustellen.

\begin{equation*}
\begin{aligned}
z_1: 140 - (140*\alpha + (1-\alpha)*90) = 50\alpha +90 \\
z_2: 120 - (80*\alpha + (1-\alpha)*120) = -40\alpha+120
\end{aligned}
\end{equation*}

Setze die daraus resultierenden Gleichungen gleich und löse nach \(\alpha\) auf um das \(\alpha\) zu erhalten, bei dem das Bedauern unabhängig vom eintretenden Umweltzustand gleich ist.

\begin{equation*}
\begin{aligned}
 50\alpha +90 = -40\alpha+120\\
90\alpha = 30\\
\alpha = \frac{1}{3}
\end{aligned}
\end{equation*}

\(\rightarrow\) Er wählt \(\alpha = \frac{1}{3}\) um das Bedauern unabhängig vom Umweltzustand zu minimieren

\subsubsection{Laplace}
\label{sec:orgc7eb33f}
Annahme, dass alle Umweltzustände mit der gleichen Wahrscheinlichkeit auftreten (bei zwei Umweltzuständen jeweils 50\%). Daher Bildung der durchschnittlich zu erwartenden Auszahlung:
\begin{equation*}
\begin{aligned}
s_1: (140 + 80) * 0.5 = 110 \\
s_2: (90+120)*0.5=105
\end{aligned}
\end{equation*}
\(\rightarrow\) Wähle s\(_{\text{1}}\) weil höherer Erwartungswert.
\newline

\textbf{Mischanlage}\\
Bilde die durchschnittlich zu erwartende Auszahlung der Mischstrategie: \(MS_{z_1}*0.5+MS_{z_2}*0.5\)

\begin{equation*}
\begin{aligned}
&0.5*(140*\alpha + (1-\alpha)*90) + 0.5*(80*\alpha + (1-\alpha)*120) \\ 
&= 0.5*(50\alpha + 90) + 0.5*(-40\alpha + 120)\\ 
&= 25\alpha + 45 - 20\alpha + 60\\
&= 5\alpha + 105
\end{aligned}
\end{equation*}

\(\rightarrow\) \(5\alpha + 105\) wird maximiert wenn \(\alpha\) = 1 ist, daher wähle \(\alpha\) = 1 was die gesamte Fokussierung auf s\(_{\text{1}}\) bedeutet.\\

\subsection{Entscheidungen unter Risiko}
\label{sec:org9967221}
Entscheidungen unter Risiko entscheiden sich von Entscheidungssituationen unter Ungewissheit insofern, dass man davon ausgehen kann, dass die Wahrscheinlichkeiten für das Eintreten bestimmter Umweltzustände als bekannt vorausgesetzt werden. \\
\subsubsection{Bayes Regel}
\label{sec:orgeda5061}
Zu den Entscheidungsregeln in solchen Szenarien zählt die \textbf{Bayes-Regel}. Diese sagt aus, dass der Entscheider sich nur nach den Erwartungswerten orientiert. Man multipliziert hierzu die Wahrscheinlichkeit des jeweiligen Zustandes mit der dazugehörigen Auszahlung und addiert dies, um den Erwartungswert für eine Strategie zu erhalten. Dabei ist zu beachten das die Auszahlung zuvor in die Nutzenfunktion des Entscheiders eingesetzt werden muss.\\

Beispiel: \emph{Die Wahrscheinlichkeit von z\(_{\text{1}}\) beträgt 75\%, von w\(_{\text{2}}\) 25\%.} 
\begin{center}
\begin{tabular}{c|c|c}
W & w\(_{\text{1}}\)=75\% & w\(_{\text{2}}\)=25\%\\
\hline
S\textbackslash{Z} & z\(_{\text{1}}\) & z\(_{\text{2}}\)\\
\hline
s\(_{\text{1}}\) & 100 & 81\\
s\(_{\text{2}}\) & 78 & 151\\
\end{tabular}
\end{center}
\textbf{Risikoneutraler Entscheider}:\\
(lineare) Nutzenfunktion \(u(\pi)=\pi\), E\(_{\text{i}}\) Erwartungswert von Strategie i

\begin{equation*}
\begin{aligned}
E(u(\pi))_{s_1} = 0.75*u(100) + 0.25*u(81) = 0.75 * 100 + 0.25*81=95.25\\
E(u(\pi))_{s_2}= 0.75*u(74) + 0.25*u(141) = 0.75*78 + 0.25*151 = 96.25
\end{aligned}
\end{equation*}
\(\rightarrow\) Wenn der Entscheider risikoneutral ist würde er sich für Strategie s\(_{\text{2}}\) entscheiden, da diese den höheren Erwartungswert hat.\\
\textbf{Risikoscheuer Entscheider}:\\
Nutzenfunktion \(u(\pi)=\sqrt{\pi}\), E\(_{\text{i}}\) Erwartungswert von Strategie i

\begin{equation*}
\begin{aligned}
E(u(\pi))_{s_1} = 0.75*u(100) + 0.25*u(81) = 0.75 * \sqrt{100} + 0.25*\sqrt{81}=9.75\\
E(u(\pi))_{s_2}= 0.75*u(74) + 0.25*u(141) = 0.75*\sqrt{78} + 0.25*\sqrt{151} = 9.70
\end{aligned}
\end{equation*}
\(\rightarrow\) Wenn der Entscheider risikoavers ist würde er sich für Strategie s\(_{\text{1}}\) entscheiden, da diese den höheren Erwartungswert hat. Grund: s\(_{\text{2}}\) hat zwar an sich den höheren Erwartungswert (siehe risikoneutral) ist aber riskanter.

\subsubsection{Sicherheitsäquivalent \& Risikoprämie}
\label{sec:org58d9c61}
Das Sicherheitsäquivalent einer unsicheren Zahlung ist der Betrag einer äquivalenten sicheren Zahlung:
\(u(SÄ) = E(u(\pi))\) . Der Wert des SÄ hängt dementsprechend direkt von der individuellen Nutzenfunktion u(\(\pi\)) des Entscheiders ab.

Beispiel: Lineare Nutzenfkt \(u(\pi)=\pi\) \\
\begin{equation*}
\begin{aligned}
E(u(\pi))_{s_1} = 95.25 == u(\textnormal{SÄ}) \rightarrow \textnormal{SÄ} = 95.25\\
E(u(\pi))_{s_2} = 96.25 == u(\textnormal{SÄ}) \rightarrow \textnormal{SÄ} = 96.25
\end{aligned}
\end{equation*}
weil \(u(\pi) == \pi\) ist \(95.25 == u(95.25)\) \\

Beispiel: Wurzel-Nutzenfkt \(u(\pi)=\sqrt{\pi}\) \\
\begin{equation*}
\begin{aligned}
E(u(\pi))_{s_1} = 9.75 == u(\textnormal{SÄ}) \rightarrow u(\textnormal{SÄ}) = 9.75 \rightarrow \sqrt{\textnormal{SÄ}}=9.75 \rightarrow \textnormal{SÄ}= 95.06\\
E(u(\pi))_{s_2} = 9.70 == u(\textnormal{SÄ}) \rightarrow u(\textnormal{SÄ}) = 9.70 \rightarrow \sqrt{\textnormal{SÄ}}=9.70 \rightarrow \textnormal{SÄ}= 94.09
\end{aligned}
\end{equation*}\\
\newline
Die Risikoprämie ist die Differenz zwischen dem Erwartungswert und dem Sicherheitsäquivalent. Sie misst wie viel dem Entscheider die Eliminierung des Risikos wert ist.\\
Beispiel: Lineare Nutzenfkt $u(\pi)=\pi$ \\
Erwartungswert entspricht dem Sicherheitsäquivalent daher ist Risikoprämie = 0.\\
Beispiel: Wurzel-Nutzenfkt $u(\pi)=\sqrt{\pi}$ \\
\begin{equation*}
\begin{aligned}
RP_{s_1}= E(\pi)_{s_1} - \textnormal{SÄ} = 95.25 - 95.06 = 0.19 \\
RP_{s_2}= E(\pi)_{s_2} - \textnormal{SÄ} = 96.25 - 94.09 = 2.16 \\
\end{aligned}
\end{equation*}

\subsubsection{Bernoulli Prinzip}
\label{sec:orgb71c908}
Sankt-Petersburg-Paradoxon stellt Bayes Regel in Frage. Es zeigt, dass die Berücksichtigung von Erwartungswerten nicht in allen Fällen dem Entscheidungsverhalten von Menschen in der Realität entspricht (Beispiel Münzwurf 1\texteuro{} wenn Kopf beim ersten Wurf, 2\texteuro{}  wenn Kopf beim zweiten Wurf, 4\texteuro{}  wenn Kopf beim dritten Wurf, 2\(^{\text{n-1}}\)\texteuro{} wenn Kopf beim n-ten Wurf \(\rightarrow\) Erwartungswert = \(\infty\) \(\rightarrow\) Individum würde sein gesamtes Vermögen einsetzen, um an der Lotterie teilzunehmen?!).\\
Das Bernoulli Prinzip löst jenes Paradoxon und gilt als rationales Entscheidungskriterium.\\
Annahme des BP: Der Spieler hat eine Wurzel-Nutzenfunktion, somit wäre der erwartete Nutzen:\\
\begin{equation*}
\begin{aligned}
E(u(\pi))=\frac{1}{2}*\sqrt{1} +\frac{1}{4}*\sqrt{2} +\frac{1}{8}*\sqrt{4} + ... \approx 1.71\\
\end{aligned}
\end{equation*}
\(\rightarrow\)  entspricht einer sicheren Zahlung (SÄ) von 2.91, da:\\
\(1.71 = \sqrt{\textnormal{SÄ}} \rightarrow 1.71 = \sqrt{2.91} \rightarrow \textnormal{SÄ} = 2.91\) \\
Somit wäre der Spieler bereit einen endlichen Betrag für die Lotterie zu zahlen, was realistischer ist.

\subsubsection{Maße für Risikoaversion}
\label{sec:orga95b79d}
Individuen mit \textbf{konkaver} (\(\frown\), zweite Ableitung < 0) Nutzenfunktion sind risikoavers. Es gilt, dass der Erwartungswert des Nutzens kleiner ist als der Nutzen des Erwartungswerts: \(E(u(\pi)) < u(E(\pi)\) .\\
Individuen mit \textbf{konvexer} (\(\smile\), zweite Ableitung > 0) Nutzenfunktion sind risikofreudig. Es gilt, dass der Erwartungswert des Nutzens größer ist als der Nutzen des Erwartungswerts: \(E(u(\pi)) > u(E(\pi)\) .\\
Individuen mit \textbf{linearer} (---) Nutzenfunktion sind risikofreudig. Es gilt, dass der Erwartungswert des Nutzens und der Nutzen des Erwartungswerts übereinstimmen.\\
Beispiel:
\(E(\pi)_{s_1} = 100 *0.4 + 60 * 0.6 = 76\) \\
\textbf{bei risikoaverser Nutzenfkt \(u(\pi)=\sqrt{\pi}\) :}\\
\begin{equation*}
\begin{aligned}
&E(u(\pi))_{s_1} = \sqrt{100} * 0.4 + \sqrt{60} 0.6 = 8.65 \\
&u(E(\pi))_{s_1}= \sqrt{E(\pi)} = \sqrt{76} =8.72 \\
&\rightarrow E(u(\pi)) < u(E(\pi) \rightarrow 8.65 < 8.72 \checkmark\\
\end{aligned}
\end{equation*}
Bedeutet dass der Entscheider eine sichere Zahlung in Höhe von E(\(\pi\)) besser als eine Lotterie mit dem gleichen Erwartungswert findet.\\
\newline
\textbf{bei risikofreudiger Nutzenfkt \(u(\pi)=\pi^2\) :} \\
\begin{equation*}
\begin{aligned}
&E(u(\pi))_{s_1} = 100^2 * 0.4 + 60^2 * 0.6 = 6160 \\
&u(E(\pi))_{s_1}= {E(\pi)}^2 = {76}^2 =5776 \\
&\rightarrow E(u(\pi)) > u(E(\pi) \rightarrow 6160 > 5776 \checkmark\\
\end{aligned}
\end{equation*}
\newline
Wie kann man die Stärke der Risikoaversion messen?
\paragraph{Absolute Risikoaversion}
\(ARA(\pi)= -\frac{u''(\pi)}{u'(\pi)}\) ist unabhängig vom Anfangsvermögen des Entscheiders (Entscheidungen sind in diesem Sinne \textbf{absolut}). Wenn ARA konstant, dann konstante \textbf{absolute} Risikoaversion.\\
\paragraph{Relative Risikoaversion}
\(RRA(\pi)= -\pi * \frac{u''(\pi)}{u'(\pi)}\) wird durch Anfangsvermögen des Entscheiders \textbf{relativiert}. Wenn RRA konstant, dann hat der Entscheider eine konstante \textbf{relative} Risikoaversion\\
\subsection{Dynamische Entscheidungen}
\label{sec:org774dc44}
Beispiel:\\
\begin{istgame}[scale=1.5,font=\footnotesize]
\xtdistance{15mm}{43mm}
\istroot(0){Anfangsknoten}
  \istb{Investition}[al]
  \istb{keine Investition}[ar]
  \endist
\xtdistance{15mm}{20mm}
\istroot(1)(0-1)<180>{}
  \istb{Marketing}[l]
  \istb{kein Marketing}[r]
  \endist
\istroot(2)(0-2)<0>{}
  \istb{Marketing}[l]
  \istb{kein Marketing}[r]
  \endist
\xtdistance{15mm}{10mm}
\istroot(a)(1-1)<180>{10}
  \endist
\istroot(b)(1-2)<0>{5}
  \endist
\istroot(c)(2-1)<180>{6}
  \endist
\istroot(d)(2-2)<0>{7}
\endist
\end{istgame}    
\newline
Lösung über Rückwärtsinduktion: Betrachte Teilbäume des eigentlichen Entscheidungsbaum und treffe Entscheidung, lasse dann die Kanten des betrachteten Teilbaums fallen und schreibe den Nutzen der gewählten Entscheidung an seinen Knoten:\\
\begin{istgame}[scale=1.5,font=\footnotesize]
\xtdistance{5mm}{20mm}
\istroot(0){Anfangsknoten}
  \istb{Investition}[al]
  \istb{keine Investition}[ar]
  \endist
\xtdistance{5mm}{10mm}
\istroot(1)(0-1)<180>{10}
  \endist
\istroot(2)(0-2)<0>{7}
  \endist
\end{istgame}    

\paragraph{Perfekte Information, aber Züge der Natur}
Ist konzeptionell genau identisch zum obigen, grundlegenden Fall(perf Inf, keine Züge der Natur). Der einzige Unterschied ist, dass die Natur zwischendurch zieht und somit Unsicherheit/Risiko ins Spiel bringt. Deshalb wendet man die Konzepte des Entscheidens unter Unsicherheit(Ungewissheit?!)/Risiko am entsprechenden Knoten an.\\
Beispiel: \emph{Regenschirmproduktion gibt bei gutem Wetter(25\%) 9 und bei schlechtem Wetter(75\%) 10. Sonnenschirmproduktion gibt respektive 11 und 8.}\\
\begin{istgame}[scale=1.5,font=\footnotesize]
\xtdistance{5mm}{20mm}
\istroot(0){}
  \istb{R}[al]
  \istb{S}[ar]
  \endist
\xtdistance{7mm}{10mm}
\istroot(1)(0-1)<180>{}
  \istb{g:\frac{1}{4}}[l]
  \istb{s:\frac{3}{4}}[r]
  \endist
\istroot(2)(0-2)<0>{}
  \istb{g:\frac{1}{4}}[l]
  \istb{s:\frac{3}{4}}[r]
  \endist
\xtdistance{15mm}{10mm}
\istroot(a)(1-1)<180>{9}
  \endist
\istroot(b)(1-2)<0>{10}
  \endist
\istroot(c)(2-1)<180>{11}
  \endist
\istroot(d)(2-2)<0>{8}
\endist
\end{istgame}    


\section{Notes}
\label{sec:orgf0c9c4c}
\begin{enumerate}
\item z \(\epsilon\) Z bedeutet alle anderen Variablen zum Beispiel s\(_{\text{1}}\) konstant zu halten und die Auszahlungen fuer alle z zu betrachten
\item ableitungsregeln, partielle ableitung, ableitung von spezialfaellen, bruch in wurzel schreiben
\end{enumerate}
\end{document}
