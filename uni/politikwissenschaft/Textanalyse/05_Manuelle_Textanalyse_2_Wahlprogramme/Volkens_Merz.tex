% Created 2018-08-29 Wed 19:28
% Intended LaTeX compiler: pdflatex
\documentclass[11pt]{article}
\usepackage[utf8]{inputenc}
\usepackage[T1]{fontenc}
\usepackage{graphicx}
\usepackage{grffile}
\usepackage{longtable}
\usepackage{wrapfig}
\usepackage{rotating}
\usepackage[normalem]{ulem}
\usepackage{amsmath}
\usepackage{textcomp}
\usepackage{amssymb}
\usepackage{capt-of}
\usepackage{hyperref}
\date{\today}
\title{}
\hypersetup{
 pdfauthor={},
 pdftitle={},
 pdfkeywords={},
 pdfsubject={},
 pdfcreator={Emacs 26.0.91 (Org mode 9.1.13)}, 
 pdflang={English}}
\begin{document}

\tableofcontents

\section{Volkens, Merz - Die Qualität von Wahlprogrammen}
\label{sec:org0fdb53c}
\subsection{Einleitung}
\label{sec:org2af14eb}
\begin{itemize}
\item Wahlprogramme besondere Bedeutung im Hinblick auf Wählerinteressen
\begin{itemize}
\item weit verbreitete Meinung, dass Parteien dem Anbieten klarer programmatischer Alternativen (zentrale Rolle im repräsentativen Prozess) nicht mehr nachkommen
\begin{itemize}
\item \(\rightarrow\) sinkende Zahlen von Parteimitgliedern, Kernwählern und sinkende Parteiidentifikation
\end{itemize}
\end{itemize}
\item Politikwissenschaft: Diagnose der Transformation zu catch-all Parteien (1960er) und Kartellparteien(1990er), so dass sich Wahlprogramme nicht mehr wirklich unterscheiden
\item Rational-Choice: Parteienpositionen nähern sich, wenn sie zwecks Stimmenmaximierung um Mehrheit der Wähler in der Mitte des pol Spektrums konkurrieren
\end{itemize}

\(\rightarrow\) \textbf{Sind die oben genannten Thesen empirisch haltbar?}

\subsubsection{Erkenntnisinteresse \& Vorgehen}
\label{sec:org5d3fac5}
\begin{enumerate}
\item Fragestellung
\label{sec:org9ef6ea6}
\begin{itemize}
\item inwieweit Parteien mit Wahlprogrammen antreten, die inhaltliche Prioritäten setzen und Alternativen bieten, die für den Bürger sichtbar und klar formuliert sind
\end{itemize}

\item Vorgehen
\label{sec:org003995b}
\begin{itemize}
\item Entwicklung eines mehrdimensionales Konzept der Qualität des programmatischen Angebots
\begin{itemize}
\item Unterscheidung in vier Qualitätsaspekten der Differenzierung von Positionen
\item Unterscheidung der Sichtbarkeit, Klarheit und Heterogenität von Prioritäten
\end{itemize}
\item quantitative Inhaltsanalyse der Wahlprogramme in Verbindung mit Wahlstatistiken
\begin{itemize}
\item Analyse der Entwicklungen der Qualität von 2103 Programmen von 279 versch. Parteien zu 371 Wahlen in 21 OECD-Ländern zwischen 1950 und 2011 im quantitativen Vergleich
\end{itemize}
\item aufgrund des langen Betrachtungszeitraums und damit einhergehenden Veränderungen des politisch-programmatischen \& gesellschaftlichen Raums \emph{keine} Betrachung auf einer einzigen abstrakten Links-Rechts-Dimension, sondern Vergleich der 4 Qualitätsaspekte des programmatischen Angebots auf 3 konkreten Konfliktdimensionen:
\item soziökonomische Dimension
\item soziokulturelle Dimension
\item Zentrum-Peripherie Dimension

\item anschließend Analyse der Qualitätsunterschiede zw Demokratietypen
\end{itemize}
\end{enumerate}

\subsection{Demokratische Norm, Parteienwettbewerb und programmatische Angebote}
\label{sec:orgab2c6cf}
Ausgangspunkt des \emph{Responsible Party Models} sind pol Parteien, die Wählern programmatische Alternativen bieten sollen (Funktionsbedingung parteipolitischer Repräsentation)
\begin{itemize}
\item dem entgegen besagen Theorien des Parteienwettbewerb, dass Parteien Strategien verfolgen, die zur Angleichung ihrer Programme führen und sie Motive haben ihre Motive zu verschleiern
\item für die westeuropäischen Länder zeigt Franzmann, dass sich die Niveaus und auch die Verläufer programmatischer Differenzierungen hinsichtlich Positionen \& Prioritäten voneinander unterscheiden
\begin{itemize}
\item der positionellen Wettbewerbstheorie zufolge erwarten die Autoren eine abnehmende Differenzierung von Positionen, wenn Parteien um den Wähler der pol Mitte kämpfen
\item der Salienztheorie zufolge erwarten die Autoren eine aufgrund des Parteienwettbewerbs beschränkte Differenzierung der Prioritäten
\item 
\end{itemize}
\end{itemize}

\subsection{Veränderungen des programmatischen Angebots auf drei Konfliktdimensionen}
\label{sec:org9ffd091}
\begin{itemize}
\item von allen methodischen Ansätzen zur Messung von Parteipositionen erbringt bislang nur die klassische quantitative Analyse von Wahlprogrammen (und der darauf beruhende Datensatz MARPOR-Projekts) die benötigten langen Zeitreihen (für Längsschnittdaten zur Programmatik der Parteien)
\end{itemize}

\subsubsection{Daten \& Operationalisierung von Parteipositionen}
\label{sec:org1d30426}
Im MARPOR-Projekt wird jedes Wahlprogramm in sogenannte Quasi-Sätze (\emph{statements}) unterteilt
\begin{itemize}
\item jedes \emph{statement} wird dann von einem geschulten Coder einer von 56 Kategorien zugeordnet
\begin{itemize}
\item diese Kategorien umfassen unterschiedliche politische Ziele wie zum Beispiel den Ausbau des Wohlfahrtsstaates, die Verbesserung des Umweltschutzes oder mehr Markregulierung
\item der Datensatz gibt Aufschluss darüber, wie häufig jedes dieser 56 Ziele in jedem der untersuchten Wahlprogramme vorkommt
\begin{itemize}
\item er umfasst derzeit 3.679 Wahlprogramme von 923 Parteien mit mindestens einem Sitz in 55 Parlamenten seit 1945 bzw. dem Zeitpunkt, an dem erste demokratische Wahlen stattgefunden haben
\end{itemize}
\end{itemize}
\end{itemize}

\subsubsection{Operationalisierung von drei Konfliktdimensionen}
\label{sec:org6206f82}
\begin{itemize}
\item deduktive Herleitung von Konfliktdimensionen
\begin{itemize}
\item geht von einer begrenzten Zahl von 1 bis 3 Dimensionen des pol Wettbewerbs aus
\end{itemize}
\item Berücksichtigung der Mehrdimensionalität der meisten Parteiensysteme
\item drei Konfliktlinien werden in zahlreichen Studien immer wieder erwähnt:
\item sozioökonomische Konfliktlinie:
\begin{itemize}
\item beschreibt traditionellen Konflikt zw Arbeit und Kapital
\begin{itemize}
\item Auseinandersetzungen zw Arbeitgebern \& Gewerkschaften, Staat \& Wirtschaft
\item Haushaltspolitik \& Wirtschaftswachstum
\end{itemize}
\end{itemize}
\item soziokulturelle Konfliktlinie:
\begin{itemize}
\item Sachfragen bzgl dem gesellschaftlichen Zusammenleben
\item progressiv-libertäre Politik vs konservativ-autoritäre Politik
\end{itemize}
\item Zentrum-Peripherie Konfliktlinie:
\begin{itemize}
\item Rolle des Nationalstaates
\item Abgabe von Souveränitätsrechten
\item Europäisierung \& Internationalisierung
\end{itemize}
\end{itemize}

\(\rightarrow\) Seite 11 Bild der Konfliktdimensionen und Codes

Identifizierung von 5 Sachfragen auf jeder Konfliktdimension die sich mit dem MARPOR-Kategorienschema operationalisieren lassen
\begin{itemize}
\item zu diesen Sachfragen berechnen die Autoren für jede Partei zu jeder Wahl eine Priorität und eine Position
\begin{itemize}
\item die Priorität wird über den Salienzwert gemessen, in dem die relativen Häufigkeiten der Kategorien zusammengezählt werden, die die Sachfrage betreffen
\item die Position wird berechnet, indem die Prozentwerte in der jeweils rechten Spalte von denen in der linken Spalte voneinander abgezogen werden und durch die Summe der beiden relativen Häufigkeiten (Salienz) teilen
\end{itemize}
\end{itemize}

\subsubsection{Die Relevanz der drei Konfliktdimensionen}
\label{sec:org6665f55}
\begin{itemize}
\item um Relevanz von Konfliktlinien in der Programmatik der Parteien zu ermitteln, wird zunächst die Zahl der \emph{statements} betrachtet, die den Konfliktbereichen gewidmet sind
\item die Relevanz wird dann als Summe der rel Häufigkeiten aller Ziele einer Dimension in einem Programm (gewichtet nach Stärke der Parteien) im Durchschnitt aller relevanten Parteien in 21 Parlamenten zw 1950 und 2011 bestimmt
\end{itemize}

Ergebnisse
\begin{itemize}
\item von den 3 Konfliktbereichen nehmen sozioökonomische Ziele mit durchschn 25\% den größten Anteil am Gesamtumfang der Wahlprogramme ein
\begin{itemize}
\item werden dauerhaft berücksichtigt
\end{itemize}
\item soziokulturelle stehen mit 15\% des durchschn Programmumfangs an zweiter Stelle
\begin{itemize}
\item werden zunehmend berücksichtigt
\end{itemize}
\item mit durchschn 10\% des Umfangs werden Zentrum-Peripherie Konflikte thematisiert
\item nur schwache Korrelation der drei Dimensionen
\end{itemize}

\subsubsection{Veränderung der Lagerung im programmatischen Raum}
\label{sec:org1c97ff2}
Seite 15 \(\rightarrow\) nicht besonders relevant daher skip 

\begin{itemize}
\item 
\end{itemize}

\subsection{Qualitätsaspekte des parteiprogrammatischen Angebots}
\label{sec:org2860b47}
\begin{itemize}
\item Konzept der programmatischen Lagerung entscheidet sich maßgeblich vom Konzept der Qualität
\end{itemize}

\subsubsection{Die Operationalisierung von vier Qualitätsaspekten}
\label{sec:orgb1be80f}
Vier Qualitätsaspekte von Parteiprogrammen:
\begin{enumerate}
\item Differenzierbarkeit
\begin{itemize}
\item Distanz zw Positionen der am meisten rechts und am meisten links stehenden Partei
\end{itemize}
\item Sichtbarkeit
\begin{itemize}
\item Distanz zwischen größter und zweitgrößter Partei
\end{itemize}
\item Klarheit
\begin{itemize}
\item Berechnung von (In)konsistenz der Positionen
\end{itemize}
\item Heterogenität
\begin{itemize}
\item durchschnittliche Varianz der Sachfragenprioritäten der Parteien bei einer Wahl auf einer Konfliktdimension geteilt durch quadrierte Anzahl der Parteien
\end{itemize}
\end{enumerate}

Betrachtung dieser Aspekte für jeweils eine Wahl und getrennt für jede Konfliktdimension und dann Forschreibung dieses Wertes für die nächsten Jahre bis zur nächsten Wahl

\subsubsection{Veränderungen der Qualität des parteiprogrammatischen Angebots (1950-2011)}
\label{sec:org50b88ef}
\begin{itemize}
\item aufgrund vorhandener Krisenliteratur zu erwartende Auswirkungen werden von den 21 OECD Ländern nur eingeschränkt erfüllt (\(\rightarrow\) siehe Text Seite 22 und Graphen)
\end{itemize}

\subsection{Qualitätsunterschiede zwischen Ländergruppen}
\label{sec:orgc2ef235}
Qualitätsaspekte nach Demokratietypen
\begin{center}
\begin{tabular}{lrrrr}
 & Differenzierung & Sichtbarkeit & Klarheit & Heterogenität\\
\hline
Majoritäre Demokratien & 0.20 & 0.15 & 0.35 & 4.90\\
Konkordanzdemokratien & 0.31 & 0.14 & 0.37 & 4.50\\
Defekte Demokratie(Türkei) & 0.16 & 0.10 & 0.25 & 4.49\\
Vergleichsgruppe & 0.33 & 0.16 & 0.35 & 4.49\\
\end{tabular}
\end{center}

\subsection{Zusammenfassung}
\label{sec:orgb21b92c}
\begin{itemize}
\item Ergebnisse liefern nur wenige Hinweise auf Verschlechterung der programmatischen Qualität
\item keine Anhaltspunkte für einen generellen Bedeutungsverlust
\item Relevanz soziokultureller Konflikte hat sogar zugenommen
\item Veränderung der Lagerung des Angebots
\item Verschlechterung der sozioökonomischen Heterogenität der Programme
\item Bürger können Wahlprogramme inziwschen kaum Prioritätensetzung mehr entnehmen
\begin{itemize}
\item im Gegenzug aber Verbesserung der Klarheit von Positionen
\item \(\rightarrow\) Ausgleich der Verschlechterung eines Aspekts durch Verbesserung eines anderen Aspekts
\end{itemize}
\item auch keine Anhaltspunkte für generelle Krise der Angebotsseite demokr Wählens bezüglich unterschiedlicher Demokratietypen
\begin{itemize}
\item majoritäre Systeme bieten dem Bürger weniger Unterschiede in sozioökonomischen Positionen als proportionale Systeme
\item Konsensdemokratien leigen auf vergleichbarem Qualitätsniveau wie Konkurrenzdemokratien
\item nur in defekter Demokratie kumulieren sich erwartungsgemäß Qualitätsdefizite
\end{itemize}
\end{itemize}

Das überwiegend positive Bild trifft allerdings nur auf das elektorale Programmangebot zu. Die Sichtbarkeit der Programmalternativen der beiden großen Parteien im Parlament ist dauerhaft sehr gering ausgeprägt
\begin{itemize}
\item die von vielen Bürgern geäußerte Kritik an Ununterscheidbarkeit der Programmangebote findet hier ihre Begründung und Berechtigung
\begin{itemize}
\item diese Kritik könnte zugenommen haben weil Klarheit des Angebots zugenommen hat
\begin{itemize}
\item denn mit zunehmender Klarheit der Positionen dürfte geringe Differenzierung zw großen Parteien für Wähler immer offensichtlicher geworden sein
\item in diesem Fall hätte ironischerweise die objektive Verbesserung des Angebotsaspektes zu einer schlechteren subjektiven Beurteilung des gesamten Angebots geführt
\end{itemize}
\end{itemize}
\end{itemize}
\end{document}
