% Created 2018-04-29 Sun 22:50
% Intended LaTeX compiler: pdflatex
\documentclass[11pt]{article}
\usepackage[utf8]{inputenc}
\usepackage[T1]{fontenc}
\usepackage{graphicx}
\usepackage{grffile}
\usepackage{longtable}
\usepackage{wrapfig}
\usepackage{rotating}
\usepackage[normalem]{ulem}
\usepackage{amsmath}
\usepackage{textcomp}
\usepackage{amssymb}
\usepackage{capt-of}
\usepackage{hyperref}
\date{\today}
\title{}
\hypersetup{
 pdfauthor={},
 pdftitle={},
 pdfkeywords={},
 pdfsubject={},
 pdfcreator={Emacs 26.0.91 (Org mode 9.1.6)}, 
 pdflang={English}}
\begin{document}

\tableofcontents

\section{Tim Spier - Populismus und Modernisierung (2006)}
\label{sec:org43d9d97}
\subsection{Begriffe}
\label{sec:orge960fef}
Die politische Agitation (lat. agitare ‚aufregen‘, ‚aufwiegeln‘) steht für:

(abwertend) die meist aggressive Beeinflussung anderer in politischer Hinsicht. Der Begriff wird in der Umgangssprache, aber auch in journalistischen Kommentaren bisweilen abwertend benutzt. Der Agitator wird oft gleichgesetzt mit einem Aufwiegler, Anstifter, Hetzer und Unruhestifter (siehe Demagoge);
politische Aufklärungsarbeit oder Werbung für politische oder soziale Ziele.
\subsection{Einleitung}
\label{sec:org315f793}
\begin{itemize}
\item Populismus ist vielseitiger, komplexer und verbreiteter als es erste Annahmen vermuten lassen
\begin{itemize}
\item um derartig unterschiedliche Entscheidungsformen unter einem Oberbegriff zu klassifizieren müssen etwaige Gemeinsamkeiten vorliegen, um dies zu Rechtfertigen
\end{itemize}
\item These des Autors: Gemeinsamkeit liegt darin, dass jene pop. Parteien und Bewegunen eine Reaktion auf Krisen im Gefolge  von gesellschaftlichen Modernisierungsprozessen darstellen
\begin{itemize}
\item gesellschaftl. Modernisierungsprozesse (und/oder gravierende ökon., kulturelle wie auch pol. Veränderungen) bringen Verwerfungen \& Umbrüche mit sich, welche in Bevölkerung Ängste \& Unsicherheit hervorrufen, die widerrum in Unzufriedenheit \& Protest umschlagen können
\begin{itemize}
\item -> bietet gute Voraussetzungen fuer pop Mobilisierung
\end{itemize}
\end{itemize}
\end{itemize}
\subsection{Populismus und Modernisierung - Zwei Seiten einer Medaille}
\label{sec:orge678627}
\subsubsection{Definition \& Wirkung von Modernisierung}
\label{sec:orgb14c809}
\begin{itemize}
\item Modernisierung = Entwicklung einer Gesellschaft von einem älteren Zustand in einen neuen
\item Populismus als Folge von Modernisierungsprozessen
\begin{itemize}
\item Konsequenzen der Moderne (pos/neg Auswirkungen) auf bestimmte Individuen/Bevölkerungsgruppen
\end{itemize}
\item Modernisierung als reine Analysekategorie, die unterschiedlichste Formen und Ausprägungen annehmen kann
\item Modernisierung ist kein neutraler oder ausschließlich positiver/negativer Prozess, da es immer Gewinner/Verlierer gibt, meist:
\begin{itemize}
\item Gewinner = Bevölkerungsgruppen, die sich an die stattfindenen Veränderungen am besten anpassen
\item Verlierer = die, die sich nicht ohne weiteres anpassen (können) oder durch den Wandel negativ betroffen sind
\end{itemize}
\item Verlierer des Modernisierungsprozesses bieten Voraussetzungen für pol Protest: allg. Unzufriedenheit -> pol Unzufriedenheit -> Politikverdrossenheit
\item schwerwiegender als Unzufriedenheit sind einhergehende psych. Probleme, die aus der sozialen Situation von Modernisierungsverlierer resultieren können (tiefgreifende Verunsicherungen)
\end{itemize}
\subsubsection{Warum profitieren ausgerechnet populistische Bewegungen von Unzufriedenheit?}
\label{sec:org14b683b}
\begin{itemize}
\item These: Populistische Agitation spricht Modernisierungsverlierer besonders an
\end{itemize}

Merkmale des Populismus:
\begin{enumerate}
\item Appell an "das Volk" oder "den kleinen Mann"
\begin{itemize}
\item das Volk (populus) wird als mehr oder weniger homogene Masse angesprochen und Unterschiede weitgehend geleugnet
\item Zuschreibung pos. Eigenschaften an das Volk zur Vermittlung eines Zugehörigkeitgefühls und einer sozialen Identität
\end{itemize}
\item Gegenüberstellung zur (homogenen) "Elite"/"Establishment", welche ebenfalls sehr klischeehaft stigmatisiert wird
\item charismatische Führerfiguren, die sich zu "Volksvertretern" hochstilisieren
\item Vornahme einer Abgrenzung von bestimmten Bevölkerungsgruppen (insb. Minderheiten)
\end{enumerate}

Einschränkung: Nicht immer folgt auf Modernisierungsprozesse ein Erfolg von Populisten, es ist ausreichendes Potential von Nöten:
\begin{itemize}
\item Verdichtung negativer Folgen von Modernisierungsprozessen zu einer krisenhaften Situation
\begin{itemize}
\item erst dieser "populistische Moment" schafft Raum zur Mobilisierung
\item schafft Gelegenheitsfenster, dass auch genutzt werden muss
\end{itemize}
\item auch etablierte Parteien können Themen aufgreifen und somit das Potential an sich binden
\item Populisten müssen ihr Potential durch Massenmedien o eigene Propaganda effektiv ausspielen
\begin{itemize}
\item effektive Hervorbringung ihrer ressentimentgeladenen Rhetorik
\end{itemize}
\item rhetorisch geschickte Führerfigur
\end{itemize}
\subsection{Historische Beispiele für den Zusammenhang von Populismus \& Modernisierung}
\label{sec:orgb34433c}
\subsubsection{Populisten Bewegung in den USA}
\label{sec:org6a69fe8}
\begin{itemize}
\item durch Industrialisierung (Ende des 19.Jhd), die viele Monopole im Nordosten hervorbrachte, fühlten sich Farmer und Siedler des Mittleren Westens und Südens zunehmend abgehängt und gerieten in Abhängigkeiten verschiedenster Natur
\item diese beklemmende wirtschaftliche Situation der Farmer mündete in \textbf{organisierten Protest}
\item Granger-Bewegung (propagierte Selbsthilfemaßnahmen) -> Greenback-Party (Abschaffung des Goldstandards; erzielte einige Wahlerfolge) -> \emph{Farmers' Alliance} (zunächst Selbsthilfestrategie, dann \textbf{aktiver politischer Protest} zB gemeinschaftlicher Boykott von Monopolkonzernen)
\end{itemize}

Demokraten \& Republikaner beachteten die Forderungen der Farmers' Alliance nicht und so wurde die \textbf{People's Party} (später auch \emph{Populist Party} genannt) gegründet um selbst bei Wahlen anzutreten
\begin{enumerate}
\item Forderungen und Erfolge der People's Party
\label{sec:orgb1072d8}
Erfolge gingen nach der Jahrhundertwende zurück, weil Protestbereitschaft der Farmer während des Konjunkturaufschwungs sank und viele pol. Forderungen von den großen Parteien aufgegriffen worden waren

\item Typische Merkmale
\label{sec:org04daf63}
Farmers' Alliance und Peoples Party wiesen wesentliche Merkmale populistischer Bewegungen auf:
\begin{itemize}
\item setzten Potential der durch Modernisierungsprozesse negativ betroffener Farmer um
\item typisch idealisiertes Volksverständnis; Beschwörung des "common man" \& hart arbeitenden Farmers; I
\begin{itemize}
\item "einfache Leute" (Süden, Mittlerer Westen) vs wirtschaftl./pol. Eliten der Ostküste
\end{itemize}
\item mehrere Agitatoren die rhetorisch begabt und mit charismatischen Qualitäten ausgestattet waren (allerdings keine überragende Führerfigur)
\item charakteristische ressentimentgeladene Abgrenzung ggü versch. Bevölkerungsgruppen (Bänker, Finanziers, Juden, Schwarze)
\begin{itemize}
\item Mischung progressiver Forderungen mit autoritären Ideologieelementen
\end{itemize}
\end{itemize}
\end{enumerate}
\subsubsection{Die Narodniki in Russland}
\label{sec:org292970a}
Narodniki (zu deutsch Volkstümer/Volksfreunde) waren eine Gruppe radikaler Intellektueller, die Unterstützung im Volk suchten
\begin{itemize}
\item eher ein Besipiel des Scheiterns einer populistischen Bewegung
\end{itemize}

Situation in Russland:
\begin{itemize}
\item ländliche Bevölkerung des 19.Jhd. in starker Abhängigkeit von Grundherren
\begin{itemize}
\item finanzielle abhängig und rechtlich benachteiligt
\end{itemize}
\item rasanter Anstieg der Bevölkerung ohne verbesserte landwirtschaftl. Erträge führte zu \textbf{Armut} und \textbf{verbitterter Stimmung}
\end{itemize}

Trotz dieses vermeindlich großen Potentials für Populismus durch Modernisierungsprozesse, blieb dieser zunächst aus. Es kam nicht zu größeren Unruhen und Organisationsversuche der Bauern waren im Wesentlichen nicht vorhanden.

In den 1870ern zogen dann Tausende junger Studenten, welche sich als Vollstrecker romantisierender Leitgedanken (Zukunft Russlands liegt bei den Bauern) intellektueller Vordenker sahen, aufs Land. Sie versuchten die Bauern vergeblich für revolutionäre Aktivitäten zu gewinnen.
\begin{itemize}
\item die städtischen Intellektuellen waren den Bauern fern/fremd
\item Sympathie für den Zaren war auf dem Land noch immer groß
\end{itemize}

Ergo führe ein "populistischer Moment" nicht automatisch zum Erfolg einer populistischen Bewegung
\subsubsection{Populismus in der Weimarer Republik}
\label{sec:orge40cb29}
\begin{enumerate}
\item Abhängung und soziale Isolierung des alten Mittelstandes
\item Händler \& Kleingewerbebetreibenden zunehmend enttäuscht von Politik und etablierten Parteien (sowie bürgerlichen Parteien)
\item Proteste durch lokale und regionale Vereinigungen, sowie spontan einberufene Versammlungen
\item Bildung von freien Fachverbänden, die nich an offizelle Dachverbände angeschlossen waren
\end{enumerate}

Bei Wahlen kam es dann vermehrt Achtungserfolgen von kleinen Splitterparteien, die sich in radikaler Weise dem Mittelstand verschrieben hatten. Anfang der 30er gelang es dann der NSDAP den alten Mittelstand zunehmend an sich zu binden.
\subsubsection{Typische Merkmale}
\label{sec:orge55705f}
\begin{itemize}
\item Beschwörung des kleinen Mannes und des gesunden Mittelstandes
\item Hochstilisierung des Mittelstandes zum ideellen Kern des Volkes
\item radikale Kritik der pol. Eliten
\item Abgrenzung ggü anderen Bevölkerungsgruppen (Unternehmer, Großkapital)
\begin{itemize}
\item Kampagnen gegen Warenhäuser und Konsumgemeinschaften, welche in Konkurrenz zu kleinen Händlern des Mittelstandes standen
\item antisemitisch aufgeladene Kampagnen gegen jüdische Warenhäuser
\end{itemize}
\item vor Hitler allerdings keine charismatische Führerfigur
\end{itemize}
\subsection{Rechtspopulismus als Schattenseite aktueller Modernisierungsprozesse}
\label{sec:orgc724736}
\begin{itemize}
\item seit Mitte der 1980er kann in vielen westlichen Industrienationen der Aufstieg "rechtspopulistischer Parteien" beobachtet werden
\item parallele Entwicklung hinsichtlich der Wahlergebnisse solcher Parteien lässt vermuten, dass es auch länderübergreifende Gründe für dieses Phänomen gibt:
\begin{itemize}
\item Phänomene ebenfalls Folge von Modernisierungsprozessen -> Prozess der Globalisierung (\emph{"Populismus als Schattenseite der Globalisierung"})
\end{itemize}
\end{itemize}

Doch inwiefern ist die Globalisierung tatsächlich für das Erstarken dieser Parteienfamilie verantwortlich?
\begin{itemize}
\item umstritten ob es "Globalisierung" überhaupt gibt
\begin{itemize}
\item wenn ja, worin liegt das spezifisch Neue der heute ablaufenden Globalisierungsprozesse
\end{itemize}
\end{itemize}
\subsubsection{Ökonomische Dimension der Globalisierung}
\label{sec:org8ca3852}
bedeutendste Dimension der Globalisierung ist die ökonomische
\begin{itemize}
\item weltweite Ausdehnung wirtschaftl Aktivitäten
\item wachsende Intensität der Waren- und Kapitalströme
\item zunehmende Exportorientierung führt dazu, dass Industrien in "Schwellenländern" in Konkurrenz zu alten Industrienationen treten
\end{itemize}

-> ökon. Globalisierung schafft Verlierer in den Bereichen der Volkswirtschaft, die von der internationalen Konkurrenz am stärksten betroffen sind und daher Strukturanpassungen vornehmen müssen
\begin{itemize}
\item Arbeitslosigkeit, Berufswechsel oder sinkende Reallöhne als mögliche Folge
\end{itemize}
\subsubsection{Kulturelle Dimension der Globalisierung}
\label{sec:org612fc6f}
\begin{itemize}
\item Zunahme der grenzüberschreitenden Kommunikation
\item Zunahme von Migration (aus Entwicklungsländern in westliche Industrieländer) und Binnenmigration
\begin{itemize}
\item Grenzen traditioneller Kulturen werden überwunden
\begin{itemize}
\item alternative Lebenstile statt konservativer, traditioneller
\end{itemize}
\end{itemize}
\end{itemize}

Aufbrechen der kulturellen Traditionen ruft bei vielen Menschen Verunsicherung hervor, sie wollen/können diese Änderungen nicht hinnehmen (diese Gruppe könnte man als Verlierer der kulturellen Globalisierung auffassen)
\subsubsection{Politische Dimension der Globalisierung}
\label{sec:org77af454}
\begin{itemize}
\item eng mit den anderen beiden Dimensionen verbunden
\item pol., ökon und ökolog Probleme lassen sich kaum noch innerhalb nationaler Politik angehen
\item Steuerungsfähigkeit des klassischen Nationalstaats nimmt tendenziell ab
\item Bürger verspüren Reduzierung demokratischer Kontroll- und Einflussmöglichkeiten
\end{itemize}
\subsubsection{Zusammenfassender Zusammenhang von Globalisierung \& Populismus}
\label{sec:orgea3ca50}
\begin{itemize}
\item \textbf{Personen die negativ von Folgen der ökon, kulturellen wie pol. Globalisierung betroffen sind bilden i.d.R den Großteil des Wählerreservoir für rechtspop. Parteien}
\begin{itemize}
\item bei diesen Personen sind häufig pol Unzufriedenheit, Statusängste, materielle Not sowie Orientierungs- und Identitätslosigkeit anzutreffen
\end{itemize}
\item Protestpotential muss dennoch effektiv angesprochen werden
\begin{itemize}
\item hierbei kommen erneut Merkmale populistischer Agitation ins Spiel:
\end{itemize}
\item charismatische Führerfigur (zB Haider, Le Pen, Fortuyn, Bossi)
\item Rekurs auf "das Volk"  und die "kleinen Leute"
\end{itemize}
3, Abgrenzung ggü pol Establishment
\begin{enumerate}
\item Abgrenzung ggü bestimmten Bevölkerungsgruppen; Appell an diesbezügliche Ressentiments (Abneigung/Voreingenommenheit)
\end{enumerate}
\subsection{Die Wähler rechtspopulistischer Parteien als Modernisierungsverlierer}
\label{sec:org0ad0324}
\begin{itemize}
\item Versuch These der Modernisierungsverlierer an empirischen Umfragedaten nachzuweisen
\item da Modernisierungsverlierer-Eigenschaft einer Person nicht einfach abfragbar ist, werden mittels Indikatoren sozialstrukturelle Merkmale \& Einstellungen betrachtet und im Lichte der Modernisierungsverlierer-These interpretiert:
\end{itemize}
\subsubsection{Indikatoren}
\label{sec:org48d70f2}
\begin{enumerate}
\item Objektive Deprivation
\item Subjektive Deprivation
\end{enumerate}

Im Falle objektiver Deprivation würde man erwarten, dass gerade jüngere männliche Industriearbeiter mit eher geringer Qualifikation betroffen, sowie Gruppe des traditionellen alten Mittelstandes aus kleinen Ladenbesitzern und Handwerken sind

Bei Indikatoren subjektiver Deprivation ist davon auszugehen, dass Wähler rechtspopulistischer Parteien eine Verschlechterung der allgemeinen wirtschaftl. Situation \& konkreten Finanzlage sehen + zusätzlich befürchten, dass sich ihre persönliche Situation weiter verschlechert

\subsection{Sitzung}
\label{sec:org4d6ce5c}
\begin{itemize}
\item Katz \& Meyer: Kartellparteithese
\item Erklärung 2018
\item Erklärungsfaktoren Arzheimer:
\begin{enumerate}
\item westl Gesellschaften eher autoritär ausgerichtet (Adorno \& co)
\item These der Modernisisierungsverlierer
\item Gruppen- und Identitätskonflikte
\item politische Kultur
\end{enumerate}
\end{itemize}
\end{document}
