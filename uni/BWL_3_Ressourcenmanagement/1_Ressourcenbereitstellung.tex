% Created 2018-06-07 Thu 20:40
% Intended LaTeX compiler: pdflatex
\documentclass[11pt]{article}
\usepackage[utf8]{inputenc}
\usepackage[T1]{fontenc}
\usepackage{graphicx}
\usepackage{grffile}
\usepackage{longtable}
\usepackage{wrapfig}
\usepackage{rotating}
\usepackage[normalem]{ulem}
\usepackage{amsmath}
\usepackage{textcomp}
\usepackage{amssymb}
\usepackage{capt-of}
\usepackage{hyperref}
\usepackage[margin=0.5in]{geometry}
\date{\today}
\title{}
\hypersetup{
 pdfauthor={},
 pdftitle={},
 pdfkeywords={},
 pdfsubject={},
 pdfcreator={Emacs 26.0.91 (Org mode 9.1.13)}, 
 pdflang={English}}
\begin{document}

\tableofcontents

\section{Ressourcen, Prozesse und Ziele betrieblicher Leistungserstellung}
\label{sec:orgd13a2d5}
\subsection{Grundbegriffe betrieblicher Leistungserstellung}
\label{sec:org8dbf7df}
\subsubsection{Produktion, Produktionsfaktoren, Produktionswirtschaft}
\label{sec:orgde801cf}
\subsubsection{Zahlungsstrom, Kapitalveränderung, Finanzwirtschaft}
\label{sec:org5c792fa}
\subsection{Gegenstandbereich \& Ziele betrieblicher Leistungserstellung}
\label{sec:org531f673}
\subsubsection{Betriebliche Leistungserstellung als Kombinationsprozess (Gutenberg)}
\label{sec:org06cae84}
"\emph{Die Ergiebigkeit des Faktoreinsatzes in den  Betrieben ist einmal von der Beschaffenheit der Faktoren selbst und zum anderen von ihrer Kombination abhängig.
Es gilt deshalb zu untersuchen, welche Umstände es sind, die den produktiven Beitrag bestimmen, den sie im Rahmen einer Faktorkombination zu leisten imstande sind.}" (Gutenberg 1975)

Beim Input der betrieblichen Leistungserstellung unterscheidet man zwischen Potential- und Repetierfaktoren
\begin{center}
\begin{tabular}{ll}
 & Repetierfaktoren (Verbrauch)\\
Charakteristik & gehen im Produktionsprozess physisch \& mengenmäßig unter\\
Bestimmung des Werteverzehrs & i.d.R leicht zu bewerten \& zuzuordnen\\
Teilbarkeit & i.d.R beliebig teilbar\\
Beispiele & Werkstoffe, Energie\\
\end{tabular}
\end{center}

\begin{center}
\begin{tabular}{ll}
 & Potentialfaktoren (Gebrauch/Bestand)\\
Charakteristik & stellen längerfristig verfügbare Nutzungspotentiale bereit\\
Bestimmung des Werteverzehrs & schwer bestimmtbar, Unsicher in der Zuordnung zB technischer Verschleiß\\
Teilbarkeit & i.d.R nicht beliebig teilbar\\
Beispiele & materiell: maschinelle Anlagen, Gebäude; immateriell: Rechte (Patente, Lizenzen), technische Informationen (Software)\\
\end{tabular}
\end{center}

\subsubsection{Ziele \& Zielkonflikte produktionswirtschaftlicher Betätigung}
\label{sec:org4a616ee}
\end{document}
