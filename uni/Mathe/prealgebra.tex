\documentclass{article}
\usepackage{mathtools}
\usepackage{amsmath, amssymb, amsthm}
\usepackage{wasysym}


\begin{document}
\part{Math Fundamentals (Pre-Algebra)}
\section{Numbers and negative numbers}

There are identity numbers for addition and  multiplication: $x+0=x$ and $x*1=x$ therefore 0 is the identity number for addition and 1 is the identity number for multiplication. Adding/multiplication by the respective identity number will always result in the origin value value.

The opposite of a number is the number multiplied by -1. An even number of negative signs is equal to a positive sign and an odd number of negative signs is equal to a negative sign: $1+--1=2$ (or $1--1=2$) and $1---1=0$. The sign of a number shows in which direction you'd have to go on the number line. Multiplication of \emph{two} numbers that have two different signs will result in a negative number, when both signs are the same the result will be positive.
Dividing a negative number by a negative number will result in a positive number, dividing a positive by a negative will result in a negative number. When more than two numbers are divided an odd number of negative signed numbers will have a negative result and an even number of negative numbers will have a positive result:
$\frac{6}{-3} = -\frac{6}{3}$ and $-\frac{-6}{3} = \frac{-1}{-1}*\frac{6}{3} = \frac{1}{1}*\frac{6}{3}$

An absolute value is referring to the distance from the origin (0): $ \left|2\right| = 2$ and $\left|-2\right| = 2$ so it will always be positive since absolute is just the units of distance.


\section{Factors and multiples}
\subsection{Divisibility}
A number is evenly divisible by 2 if the last number of it is even ($0,2,4,6,8$):
$120\div2 = 60, 126\div2=63, 128\div2=64$ 

To find out if a number is evenly by 3 you have to add all of it individual numbers together and see if that is divisible by 3: $120\div3=40$ is evenly divisible because: $1+2+0 = 3 $ 


To see if a number is evenly by 4 you have to check if the last two numbers are evenly divisble by 4: $120\div4=30$ is evenly divisible because: $20\div4=5$ 

To see if a number is evenly by 5 you have to check if the last number is either 0 or 5: $120\div5=24$ is evenly divisible by 5 because the last number is a 0, $126\div5$ is not.

The divisibility rule for 6 is achieved by testing if the number is divisble by 2 \emph{and} 3: $120\div6=20$

To find out if a number is evenly divisble by 7 you have to multiply the last number of it by 5 and then add the result to the rest of the numbers and check if that is evenly divisble by 7. $120: 0*5=0+12=12\div{7}$\lightning(not even)
$126: 6*5=30+12=42\div{7}=6$\checkmark

The divisibility for 8 is more attractive for larger numbers because you have to check the last 3 digits and see if they as a whole number are evenly divisible by 8: $120\div{8}=15$\checkmark

For 9 you sum up all the individual digits and if the sum is divisble by 9 then the whole number is divisible by 9. For 120: $1+2+0=3\div{9}$\lightning, for 126: $1+2+6=9\div{9}=1$\checkmark

A number is evenly divisible by 10 if it ends with 0.
\par

The multiples of a number are just the numbers that can be evenly divided by the number. To find for example the first 3 multiples of a number just multiply the number with 1, with 2, and with 3.

\subsection{Prime and composite numbers}




  
\end{document}