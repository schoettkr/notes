\documentclass{article}
\usepackage{mathtools}
\usepackage{amsmath, amssymb, amsthm}
\usepackage{wasysym}
\usepackage{qtree}


\begin{document}
\part{Math Fundamentals (Pre-Algebra)}
\section{Numbers and negative numbers}

There are identity numbers for addition and  multiplication: $x+0=x$ and $x*1=x$ therefore 0 is the identity number for addition and 1 is the identity number for multiplication. Adding/multiplication by the respective identity number will always result in the origin value value.

The opposite of a number is the number multiplied by -1. An even number of negative signs is equal to a positive sign and an odd number of negative signs is equal to a negative sign: $1+--1=2$ (or $1--1=2$) and $1---1=0$. The sign of a number shows in which direction you'd have to go on the number line. Multiplication of \emph{two} numbers that have two different signs will result in a negative number, when both signs are the same the result will be positive.
Dividing a negative number by a negative number will result in a positive number, dividing a positive by a negative will result in a negative number. When more than two numbers are divided an odd number of negative signed numbers will have a negative result and an even number of negative numbers will have a positive result:
$\frac{6}{-3} = -\frac{6}{3}$ and $-\frac{-6}{3} = \frac{-1}{-1}*\frac{6}{3} = \frac{1}{1}*\frac{6}{3}$

An absolute value is referring to the distance from the origin (0): $ \left|2\right| = 2$ and $\left|-2\right| = 2$ so it will always be positive since absolute is just the units of distance.


\section{Factors and multiples}
\subsection{Divisibility}
A number is evenly divisible by 2 if the last number of it is even ($0,2,4,6,8$):
$120\div2 = 60, 126\div2=63, 128\div2=64$ 

To find out if a number is evenly by 3 you have to add all of it individual numbers together and see if that is divisible by 3: $120\div3=40$ is evenly divisible because: $1+2+0 = 3 $ 


To see if a number is evenly by 4 you have to check if the last two numbers are evenly divisble by 4: $120\div4=30$ is evenly divisible because: $20\div4=5$ 

To see if a number is evenly by 5 you have to check if the last number is either 0 or 5: $120\div5=24$ is evenly divisible by 5 because the last number is a 0, $126\div5$ is not.

The divisibility rule for 6 is achieved by testing if the number is divisble by 2 \emph{and} 3: $120\div6=20$

To find out if a number is evenly divisble by 7 you have to multiply the last number of it by 5 and then add the result to the rest of the numbers and check if that is evenly divisble by 7. $120: 0*5=0+12=12\div{7}$\lightning(not even)
$126: 6*5=30+12=42\div{7}=6$\checkmark

The divisibility for 8 is more attractive for larger numbers because you have to check the last 3 digits and see if they as a whole number are evenly divisible by 8: $120\div{8}=15$\checkmark

For 9 you sum up all the individual digits and if the sum is divisble by 9 then the whole number is divisible by 9. For 120: $1+2+0=3\div{9}$\lightning, for 126: $1+2+6=9\div{9}=1$\checkmark

A number is evenly divisible by 10 if it ends with 0.
\par

The multiples of a number are just the numbers that can be evenly divided by the number. To find for example the first 3 multiples of a number just multiply the number with 1, with 2, and with 3.

\subsection{Prime and composite numbers}
A prime number is a number that is evenly divisble by 1 and iteself only. Composite numbers are all other numbers so a number is either prime or composite.
For example $1$ and $5$ both go into $5$ but not any other number between these two. And numbers higher than $5$ would return a decimal, therefore $5$ is a prime number.
For 35 there are other factors then $1$ and the number itself ($35$), e.g. $5$ or $7$ so it is an composite number.

\subsubsection{Prime Factorization}
$a * b$ $a$ and $b$ are factors (a = Multiplikator, b = Multiplikand). Prime factorization is the representation of a number $n$ as a product of prime numbers. For example:
$102 : 2 = 51$ Now 2 cant be factorized further but $51$ can be seperated into factors of $3$ and $17$. Now $3$ is (like $2$) a prime number and cant be factorized any furhter, the same goes for $17$. Now to find the product of primes you have to take all the primes and multiply them together: $2*3*17=102$.
No matter what the number is, the steps are break down the number into irreduceable prime numbers:

\Tree[.84 [.\textbf{2} ]
[.42 [.\textbf{2} ] [.21 [.\textbf{3} ] [.\textbf{7} ]
]]]

And then multiply these prime numbers: $2*2*3*7 = 2^2*3*7 = 84$

\subsection{Least common multiple}
The least common multiple of two numbers is the smallest number that is a multiple of both: $10*3=30$ and $15*2=30 \rightarrow30$ would be the LCM for 10 and 15.
To find the least common multiple of a set of numbers you have to factorize the numbers into their primes, find the highest of each factor and mulitplies these together:

\Tree[.24 [.\textbf{2} ]
[.12 [.\textbf{2} ] [.6 [.\textbf{2} ] [.\textbf{3} ]
]]]
\Tree[.10 [.\textbf{2} ] [.\textbf{5} ]]
\Tree[.15 [.\textbf{3} ] [.\textbf{5} ]]

$24=2*2*2*3=2^3*3\rightarrow 2^3$ is the largest of all factors of $2$

$10=2*5$

$15=3*5$

The factors of $3$ and $5$ are all equal so I have to take one of each: $2^3*3*5=120\rightarrow$ is the least common multiple of 24, 10 and 15.

\section{Decimals}
\subsection{Place-value notation (Stellenwertsystem) and expanded notation}
Describing a place value to the left of the decimal point, is talking about the 1s place, 10s place or the hundreds place and so on (Einerstelle, Zehnerstelle, etc). Talking about a number to the right of the decimal point is about the tenth place, the one hundredth place, the 1000th place and so on (Zehntel, Hundertstel, Tausendstel).

To write a number in expanded notation you take each number and multiply it by its corresponding place value e.g: $500 = 5*100+0*10+0*1$ and $1281=1*1000+2*100+8*10+1*1$
Of course the same thing is applicable to decimals (values to the right of the decimal point): $0.55 = 5*\frac{1}{10}+5*\frac{1}{100}=\frac{5}{10}+\frac{5}{100}$. In contrast to whole numbers where each number is multiplied by its place ($50=5*10$) with decimals you divide each number by its place ($0.5 = \frac{5}{10}$)

\subsection{Decimal arithmetic}
To add decimals together the decimals have to line up with each other:

$4.5+3.34 = 
\begin{array}[t]{r}
    4.50 \\
+ \ 3.34 \\ \hline
    7.84
\end{array}$
If the numbers do not align a $0$ can be added at the missing place behind the decimal point.

The same thing goes for subtraction:

$16.7-2.26=
\begin{array}[t]{r}
  16.70 \\
- \ 2.26 \\ \hline
  14.44
\end{array}$

\section{Fractions}


\end{document}