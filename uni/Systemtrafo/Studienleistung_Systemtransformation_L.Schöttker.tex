\documentclass[12pt]{article}
\usepackage[utf8]{inputenc}
\addtolength{\topmargin}{-.875in}
\addtolength{\textheight}{1.75in}


% \documentclass[12pt]{report}
\begin{document}
\thispagestyle{plain}
\begin{center}
    \LARGE
    \textbf{Studienleistung Systemtransformation}
    

    \small(07.12.2017)

    \vspace{0.2cm}
    \large
    Lennart Schöttker, 3151320

\end{center}
 \vspace{0.4cm}

\section{Die wichtigsten Merkmale und Folgen der chilenischen Enklavendemokratie seit 1989}
Zunächst ist festzuhalten, dass Chile, vor der von Pinochet geführten Militärdiktatur, schon einige längerfristige Demokratieerfahrungen sammeln konnte. Diese Tendenzen nahmen jedoch 1973 durch den Militärputsch ihr jähes Ende und das demokratische System wurde durch ein von Repressionen geprägtem Militärregime abgelöst.

Anfängliche wirtschaftliche Erfolge festigten die Zustimmung und gaben dem Regime eine ökonomische Legitimation, so dass es massive soziale Proteste (1982/83) und eine tiefe Wirtschaftskrise unbeschadet überstehen konnte.

Die Transition wurde mehr oder weniger von den alten Regimeeliten, mittels eines Referendums bei dem sie unterlagen, selbst eingeleitet und Pinochet nutzte seine Handlungsmacht in seiner verbleibenden Zeit, um autoritäre Enklaven zu etablieren.

Dies gelang ihm auch, da die Opposition nur begrenzte Einspruchsmöglichkeiten besaß und sie darüber hinaus nicht den Transitionsprozess gefährden wollte. Aufgrund dieses sogesehenem Subsystems innerhalb der Demokratie, welche zunächst tatsächlich eine defekte Demokratie war, spricht man von einer Enklavendemokratie.
Somit besaß das rechte Lager weit über die Lebenszeit des Regimes hinaus, unverhältnismäßige Macht:
\begin{itemize}
  \item  Sperrminoritäten
  \item  starke Eigenkompetenzen
  \item  politischer Einfluss durch Verbindungen zu Mitgliedern diverser Organe
\end{itemize}

Dies hatte zur Folge, dass ein politisches regieren \emph{gegen} das rechte Lager, mit seiner priveligierten Position, nahezu aussichtslos war.

Die Institutionalisierung der chilenischen Demokratie unterteilte sich somit in zwei Prozesse: Einerseits die Abkehrung von der Autokratie hin zur Demokratie und andererseits die Beseitigung der autoritären Überbleibsel.

Fünfzehn Jahre nach der Transition ist die Konsolidierung in Chile weit fortgeschritten und zwischen 2004 und 2005 wurden die, zu diesem Zeitpunkt fast schon irrelevanten, Enklaven durch Verfassungsreformen eliminiert.

\section{Mit welcher Transformationstheorie lässt sich die Entstehung der chilenischen Enklavendemokratie seit 1989 am besten erklären und warum?}

Die Entstehung der chilenischen Enklavendemokratie lässt sich meiner Meinung nach am besten mit der Modernisierungstheorie und dem rational-choice Ansatz erklären. 
Zum einen erlebte Chile während der Herrschaft des autoritären Regimes einen wirtschaftlichen Aufschwung.
Die Wirtschaft war für ein lateinamerikanisches Land relativ stabil und hielt sogar einer der schwersten Wirtschaftskrisen stand.
Dies spricht für die Modernisierungstheorie nach Lipset, welcher eine klare Wechselbeziehung zwischen der sozioökonomischen Entwicklungsstufe und der Demokratiefähigkeit eines Landes sieht.

Im Fall von Chile kann man zumindest davon ausgehen, dass das wirtschaftliche Wohlergehen die Entwicklungstendenzen hin zur Demokratie und die Akzeptanz gestärkt hat.

Nichtsdestotroz wurde die Transition durch die alten Regimeeliten eingeleitet und zeichnete sich durch einen schrittweise, ausgehandelten Systemwechsel aus.
Hier greift meiner Ansicht nach der rational-choice Ansatz, denn der entscheidene Schritt zur Demokratie wurde, durch das Referendum, welches über eine zweite Amtszeit Pinochets entscheiden sollte, getätigt.
Unter Angesicht der Tatsache, dass dieser, zur Überraschung des Regimes, unterlag, kann man davon ausgehen, dass sich das Regime unteranderem mit exogenem Druck konfrontiert sah. Außerdem spricht für den rational-choice Ansatz, dass die Kräfte der demokratischen Opposition und das alte Regime, sich auf politische Pakte einigten, trotz der Tatsache, dass die demokratische Opposition nur begrenzte Einwirkungsmöglichkeiten hatte, die Diskussionen nicht abbrach, um die Transition hin zur Demokratie nicht zu gefährden.

\end{document}
    