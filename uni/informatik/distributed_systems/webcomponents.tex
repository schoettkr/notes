% Created 2020-02-04 Tue 10:03
% Intended LaTeX compiler: pdflatex
\documentclass[11pt]{article}
\usepackage[utf8]{inputenc}
\usepackage[T1]{fontenc}
\usepackage{graphicx}
\usepackage{grffile}
\usepackage{longtable}
\usepackage{wrapfig}
\usepackage{rotating}
\usepackage[normalem]{ulem}
\usepackage{amsmath}
\usepackage{textcomp}
\usepackage{amssymb}
\usepackage{capt-of}
\usepackage{hyperref}
\usepackage{listings}
\linespread{1.0}
\usepackage[left=1.5cm,right=1.5cm,top=1.5cm,bottom=1.5cm]{geometry}
\setlength{\parindent}{0in}
\setlength{\parskip}{0.15cm}
\author{eo shiru}
\date{\today}
\title{}
\hypersetup{
 pdfauthor={eo shiru},
 pdftitle={},
 pdfkeywords={},
 pdfsubject={},
 pdfcreator={Emacs 28.0.50 (Org mode 9.3)}, 
 pdflang={English}}
\begin{document}

\tableofcontents


\section{Web Components}
\label{sec:orgf7e4f95}
\begin{itemize}
\item main features: HTML imports, HTML template, custom elements, shadow DOM
\end{itemize}

\section{Elements}
\label{sec:org01bf604}
\subsection{Defining a new element}
\label{sec:orgf70d54a}
The \texttt{customElements global is used for defining a custom element and teaching the browser about a new tag. Call =customElements.define()} with the tag name you want to create and a JavaScript \texttt{class} that extends the base \texttt{HTMLElement}.
\begin{itemize}
\item example: defining a mobile drawer panel, \texttt{<app-drawer>}
\lstset{breaklines=true,language=js,label= ,caption= ,captionpos=b,numbers=none}
\begin{lstlisting}
class AppDrawer extends HTMLElement {
  constructor() { // constructor arguments can also be defined
    super(); // always call super() first
    // click listener on <app-drawer> elemnt itself
    this.addEventListener('click', e => {
      if (this.disabled) {
        return;
      }
      this.toggleDrawer();
    });
  }

  // A getter/setter for on 'open' property
  get open() {
    return this.hasAttribute('open');
  }
  set open(val) {
    if (val) {
      this.setAttribute('open', ''); // refelect prop as an HTML attr
    } else {
      this.removeAttribute('open');
    }
  }
  get disabled() {
    return this.hasAttribute('disabled');
  }
  set disabled(val) {
    if (val) {
      this.setAttribute('disabled', ''); // refelect prop as an HTML attr
    } else {
      this.removeAttribute('disabled');
    }
  }

  toggleDrawer() {
    ... 
  }
}

window.customElements.define('app-drawer', AppDrawer);
\end{lstlisting}
\item the custom element created above can now be used just like native HTML elements i.e. \texttt{<app-drawer></app-drawer>}
\begin{itemize}
\item instances of it can be declared on the page, created dynamically via JS, event listeners can be attached an so on
\end{itemize}
\item \texttt{this} inside a class definition refers to the DOM itself
\begin{itemize}
\item the entire DOM API is available inside the element code for example \texttt{this.children} to inspect its direct children or \texttt{this.querySelectorAll('.items')} to query nested nodes
\end{itemize}
\end{itemize}
\subsubsection{Naming rules}
\label{sec:org553cee5}
\begin{itemize}
\item names of custom elements must contain a dash "-"
\item the same name can only be registered once
\item custom elements cannot be self-closing
\end{itemize}
\subsection{Custom element reactions}
\label{sec:org5e9723b}
A custom element can define special lifecycle hooks for running code during interesting times of its existence, these are called custom element reactions
\begin{center}
\begin{tabular}{p{6cm}p{10cm}}
Name & Called when\\
\hline
constructor & instance of the element is created or upgraded; useful for initializing state, setting up event listeners or creating a shadow dom\\
connectedCallback & called everytime the element is inserted into the DOM; useful for running setup code, such as fetching resources or rendering\\
disconnectedCallback & called everytime the element is removed from the DOM\\
attributeChangedCallback(attrName, oldVal, newVal) & called when an observed attribute has been added, removed, updated or replaced; also called for initial values when an element is created/upgraded; only attributes listed in the observerdAttributes property will receive this callback\\
adoptedCallback & the custom element has been moved into a new document\\
\end{tabular}
\end{center}
\begin{itemize}
\item to the above example \texttt{static get observedAttributes() \{ return ['disabled', 'open']\}} needs to be added to the class to have \texttt{attributeChangedCallback} called for changes in those attributes
\end{itemize}
\subsection{Creating an element that uses Shadow DOM}
\label{sec:orgd1cb272}
The Shadow DOM provides a way for an element to own, render and style a chunk of DOM that's separate from the rest of the page. You could for example hide away an entire within a single tag:
\lstset{breaklines=true,language=js,label= ,caption= ,captionpos=b,numbers=none}
\begin{lstlisting}
// chat app's implementation details are hidden away in Shadow DOM
<chat-app></chat-app>
\end{lstlisting}
To use Shadow DOM in a custom element, call \texttt{this.attachShadow} inside the constructor:
\lstset{breaklines=true,language=js,label= ,caption= ,captionpos=b,numbers=none}
\begin{lstlisting}
// Create template in js
let tmpl = document.createElement('template'); 
tmpl.innerHTML = `
  <style>:host { ... }</style> <!-- look ma, scoped styles -->
  <b>I'm in shadow dom!</b>
  <slot></slot>
`;
// or via HTML template tag
// <template id="shopping-template">
//   <b>I'm in shadow dom</b>
//   <slot></slot>
// </template>

customElements.define('x-foo-shadowdom', class extends HTMLElement {
  constructor() {
    super(); // always call super() first in the constructor.

    // Attach a shadow root to the element.
    let shadowRoot = this.attachShadow({mode: 'open'});
    shadowRoot.appendChild(tmpl.content.cloneNode(true));
  }
  ...
});
\end{lstlisting}
Example usage:
\lstset{breaklines=true,language=js,label= ,caption= ,captionpos=b,numbers=none}
\begin{lstlisting}
<x-foo-shadowdom>
  <p><b>User's</b> custom text</p>
</x-foo-shadowdom>

<!-- renders as -->
<x-foo-shadowdom>
  #shadow-root
    <b>I'm in shadow dom!</b>
    <slot></slot> <!-- slotted content appears here -->
</x-foo-shadowdom>
\end{lstlisting}
\lstset{breaklines=true,language=HTML,label= ,caption= ,captionpos=b,numbers=none}
\begin{lstlisting}
<br>
<h2>This is the footer.</h2>
<p>You can put stuff here.</p>
<br>
\end{lstlisting}
\end{document}
