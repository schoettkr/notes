% Created 2018-05-07 Mon 21:20
% Intended LaTeX compiler: pdflatex
\documentclass[11pt]{article}
\usepackage[utf8]{inputenc}
\usepackage[T1]{fontenc}
\usepackage{graphicx}
\usepackage{grffile}
\usepackage{longtable}
\usepackage{wrapfig}
\usepackage{rotating}
\usepackage[normalem]{ulem}
\usepackage{amsmath}
\usepackage{textcomp}
\usepackage{amssymb}
\usepackage{capt-of}
\usepackage{hyperref}
\date{\today}
\title{}
\hypersetup{
 pdfauthor={},
 pdftitle={},
 pdfkeywords={},
 pdfsubject={},
 pdfcreator={Emacs 26.0.91 (Org mode 9.1.6)}, 
 pdflang={English}}
\begin{document}

\tableofcontents

\section{Populism: corrective and threath to democracy (Mudde, Kaltwasser)}
\label{sec:org47c3ff4}
\subsection{Einleitung}
\label{sec:org25977be}
Analyse von 8 überregionalen Fallstudien 
\begin{itemize}
\item Populismus stehe in Konflikt mit (liberaler) Demokratie
\item wird der Einfluss von Populismus auf Demokratie durch regionale oder generelle Faktoren bzw Kontext bestimmt?
\begin{itemize}
\item hängt Vorhandensein von Populismus mit Faktoren zusammen, die in bestimmten Regionen der Welt häufiger anzutreffen sind?
\end{itemize}
\end{itemize}
\subsection{Rückgriff auf Konzept und Hypothesen und Rückschlüsse}
\label{sec:org27d7c12}
\begin{itemize}
\item minimale Definitionen/Konzepte von Demokratie \& Populismus, um konzeptuellen Geltungsbereich zu erweitern
\begin{itemize}
\item Nachteil, dass sie nicht genug Tiefe haben, um relevante Aspekte/Merkmale eines Phänomen zu unterscheiden
\end{itemize}
\item "das Volk" wird in wohlhabenden Gesellschaften eher ethnisch definiert und in ärmeren Gesellschaften eher sozio-ökonomisch
\item Populismus lässt wenig Raum für Pluralismus und öffentliche Kritik / pol Wettbewerb
\item Populisten beziehen sich häufig auf 2 Demokratiedimensionen:
\begin{itemize}
\item Kritik an schlechten \textbf{Ergebnissen} des Regimes
\item Vorlagen für Änderungen an demokr \textbf{Verfahren} (Prozesse/Prozere/Abläufen)
\item das erklärt warum sie in der Theorie Plesbizite und Formen direkter Demokr bevorzugen
\end{itemize}
\item Populismus ist nicht gegen repräsentative Demokratie, sondern appelliert für extreme Demokratie, indem sie allen ungewählten Körperschaften/Institutionen ggü enorm kritisch sind
\end{itemize}

\begin{center}
\begin{tabular}{c|c|c}
Demokratie \textbackslash{Populismus} & in Opposition & in Regierung\\
\hline
konsolidiert & kein großer Einfluss auf Demokr.qualität & \\
nicht konsolidiert & kein großer Einfluss auf Demokr.qualität & gemischte Ergebnisse (demokr Breakdown und pos Effekte)\\
\end{tabular}
\end{center}

\subsection{Von der Empirie zur Theorie: Unerwartete Befunde}
\label{sec:org4ecf0ef}
Einsichten in unerwartete Bereiche:
\subsubsection{Populismus auf subnationaler(lokaler) Ebene}
\label{sec:org69dcbc8}
\begin{itemize}
\item in Mexiko hatte Populist Lopez Obrador auf lokaler Ebene (Mexiko City) positive Wirkung, negative dann während des Präsidentenwahlkampfes (Korruption, Blockaden)
\item Haiders negative Bestrebungen (Ausgrenzung von Sloveniern) auf lokaler Ebene in Carinthia wurden durch Verfassungsgericht begraben\\
\end{itemize}
\(\rightarrow\) Populismus an der Macht auf lokaler Ebene wenig einflussreich auf Qualität der Demokratie (bzw weniger als auf nationaler Ebene)
\begin{itemize}
\item weil auf regionaler Ebene immernoch die nationale Ebene übergeordnet ist
\item und weil die regionale Ebene noch von nationaler Ebene entfernt ist begrenzen sich die Populisten und halten sich zurück (um mehr Stimmen abzufangen)
\end{itemize}
\subsection{Reaktionen auf Populismus}
\label{sec:orgaf487ea}
Herauskristillasierung von vier Strategien (treten in gemischter und nicht in Reinform auf)
\begin{itemize}
\item Isolation
\begin{itemize}
\item Leugnung jeglicher Legitimät der populistschen Behauptungen
\item Dämonisierung der Populisten
\end{itemize}
\item Konfrontation
\begin{itemize}
\item nicht nur Abstreiten der Behauptungen sondern auch Attackierung der Populisten
\end{itemize}
\item Adaption
\begin{itemize}
\item gewisse Legitimät der Forderungen der Populisten wird eingeräumt
\item Populismus in gewisser Weise eher als Mittel zur Verbesserung
\item Anregung evtl. Lernprozesse, durch die etablierte Parteien sich erneuern \& anpassen
\end{itemize}
\item Sozialisierung
\begin{itemize}
\item komplementär zur Adaption
\item kurz und langristige Maßnahmen zur Aufnahme von Populisten
\item "Pazifismus durch Deradikalisierung"
\end{itemize}
\end{itemize}
\end{document}
